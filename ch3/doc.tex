\section{Свойства преобразования Фурье}

В этом разделе мы опишем основные свойства преобразования~Фурье и докажем наиболее интересные из них.
Прежде всего, напомним внешний вид преобразования:
$$
        F[f] (\lambda) = \int\limits_{-\infty}^{+\infty}f(t)e^{-i\lambda t}\, dt,
$$
где $f \in \mbox{L}_1(-\infty,\,+\infty)$, то есть функция~$f$ интегрируема по Риману (Лебегу) на всей числовой прямой и выполнено условие
$$
        \int\limits_{-\infty}^{+\infty}|f(t)|\,dt < +\infty. 
$$

\begin{remark}
        Принадлежность функции $f$ классу $\mbox{L}_1$ гарантирует существование ее преобразования~Фурье~$F[f]$.
\end{remark}

Для начала выпишим свойства, которые напрямую следуют из определения: линейность, масштабируемость и сдвиг. Мы не будем долго на них останавливаться.
\begin{enumerate}
        \item \textbf{Линейность.}
$$
        F[\alpha f_1 + \beta f_2]
=
        \alpha F[f_1] + \beta F[f_2],
\qquad
        \forall f_1,f_2 \in \mbox{L}_1,\,\forall \alpha,\beta \in \SetR.
$$
        \item \textbf{Сдвиг.}
$$
        F[f(t - t_0)] = e^{-\lambda t_0}\cdot F[f],
$$
$$
        F[e^{i\lambda_0 t}\cdot f(t)] = F[f]\cdot(\lambda - \lambda_0).
$$
        \item \textbf{Масштабируемость.}
$$
        F[f(\alpha t)](\lambda) = \frac{1}{|\alpha|}F[f(t)]\left(\frac{\lambda}{\alpha}\right),
\qquad
        \forall \alpha \in \SetR,\,\alpha \neq 0.
$$
        \item \textbf{О четности.} Если функция~$f$ является четной, то ее образ~$F[f]$ будет действительной функцией. Если же $f$~--- нечетная, то образ~$F[f]$ будет чисто мнимой функцией.
\end{enumerate}

Теперь перейдем к более интересным свойствам. Далее каждая теорема, следствие или замечание будут являться свойствами преобразования~Фурье. Большая часть из них будет доказана.

\begin{theorem}
        Рассмотрим последовательность функций из класса $\mathrm{L}_1$, стремящююся по норме $\mathrm{L}_1$ к некоторой функции $f$ из того же класса, то есть
$$
        \{f_n\}_{n=1}^{\infty},\;f_n \in \mathrm{L}_1(-\infty,\,+\infty)
\quad : \quad
        f_n \xrightarrow[n\to\infty]{\mathrm{L}_1} f \in \mathrm{L}_1.
$$
Тогда
$$
        F[f_n] \rightrightarrows F[f].
$$
\end{theorem}
\begin{proof} Приведем несложные выкладки:
\begin{multline*}
        \sup\limits_{\lambda}|F[f_m](\lambda) - F[f_n](\lambda)| = \\
        =\sup\limits_\lambda\left|\int\limits_{-\infty}^{+\infty}(f_m(t) - f_n(t))e^{-i\lambda t}\, dt\right| \leqslant \\
        \leqslant\int\limits_{-\infty}^{+\infty}|f_m(t) - f_n(t)|\,dt < \varepsilon.
\end{multline*}
\end{proof}

\begin{theorem}
        Преобразование Фурье $F[f]$ есть непрерывная ограниченная функция.
\end{theorem}
\begin{proof}
        На самом деле ограниченность мы нечаянно вывели в предыдущей теореме. Действительно,
$$
        |F[f](\lambda)| = \left|\int\limits_{-\infty}^{+\infty} f(t) e^{-i\lambda t}\,dt\right| \leqslant \int\limits_{-\infty}^{+\infty}|f(t)|\,dt = \const.
$$
        
        С непрерывностью дела обстоят куда сложнее. Здесь нам придется записать наше преобразование в виде
$$
        F[f](\lambda) = \int\limits_{-\infty}^{+\infty}f(t)\cos(\lambda t)\,dt - i\int\limits_{-\infty}^{+\infty}f(t)\sin(\lambda t)\,dt
$$
        и сослаться на книгу А.~М.~Тер--Крикорова, М.~И.~Шабунина <<Курс математического анализа,>> где на 645 странице доказана непрерывность каждого из кусочков.
\end{proof}
\begin{remark}
        Из последней теоремы следует, например, что
$$
        F[f](\lambda)\xrightarrow[|\lambda|\to\infty]{} 0.
$$
\end{remark}

Теперь рассмотрим специальный вид функций, который часто встречается на практике непрерывные и дифференцируемые функции.

\begin{theorem}
        Пусть функция $f$ непрерывно дифференцируема, абсолютно интегрируема, и ее производная так же абсолютно интегрируема, то есть\footnote{Теорема ходит в интернете в нескольких вариантах условий: совершенно не понятно, $f$ или $f'$ должна быть непрерывной или интегрируемой. Причем доказательства везде примерно одинаковые. Здесь приведен вариант к.ф.-м.н. доцента И.~В.~Рублева.}
$$
        f \in \mathrm{C}^1(-\infty,\,+\infty)\,\cap\,\mathrm{L}_1(-\infty,\,+\infty),\;f' \in \mathrm{L}_1(-\infty,\,+\infty)
$$
        Тогда
$$
        F[f'](\lambda) = i\lambda \cdot F[f](\lambda).
$$
\end{theorem}
\begin{proof}
        Предствавим функцию в виде
$$
        f(t) = f(0) + \int\limits_{0}^{t} f'(t)\,dt.
$$
        Из сходимости интеграла $\int_0^{+\infty}f'(t)\,dt$ следует существование пределов $\lim_{t\to+\infty}f(t)$ и $\lim_{t\to-\infty}f(t)$. Они не могут быть отличными от нуля в силу сходимости интеграла $\int_{-\infty}^{+\infty}|f(t)|\,dt$. С помощью интегрирования по частям получаем
\begin{multline*}
        F[f'](\lambda) = \frac{1}{2\pi}\int\limits_{-\infty}^{+\infty}f'(t)e^{-i\lambda t}\,dt=\\
        =
        \left.\frac{1}{\sqrt{2\pi}}f(t)e^{-i\lambda t}\right|_{-\infty}^{+\infty}
        +
        \frac{i\lambda}{\sqrt{2\pi}}\int\limits_{-\infty}^{+\infty}f'(t)e^{-i\lambda t}\,dt
        = i\lambda\cdot F[f](\lambda).
\end{multline*}
\end{proof}
\begin{remark}
        Как следствие, получаем более занятную формулу:
$$
        \mbox{Пусть } f \in \mathrm{C}^{k-1}(-\infty,\,+\infty),\quad \exists f^{(k)}:f^{(k)}\in\mathrm{L}_1(-\infty,\,+\infty), \mbox{ тогда} 
$$
$$
        F[f^{(k)}](\lambda) = (i\lambda)^k\cdot F[f].
$$
\end{remark}

\begin{theorem}
        Пусть функция $f$ непрерывно дифференцируема, абсолютно интегрируема, и ее производная так же абсолютно интегрируема, то есть
$$
        f \in \mathrm{C}^1(-\infty,\,+\infty)\,\cap\,\mathrm{L}_1(-\infty,\,+\infty),\;f' \in \mathrm{L}_1(-\infty,\,+\infty)
$$
        Тогда
$$
        |F[f](\lambda)| \leqslant \frac{C}{|\lambda|}
$$
\end{theorem}
\begin{proof}
$$
        \left|\int\limits_{-T}^{+T} f(t) e^{-i\lambda t}\,dt\right|
        = \left.\frac{f(t)e^{-i\lambda t}}{-i\lambda}\right|_{-T}^{+T} + \frac{1}{\lambda}\int\limits_{-T}^{+T} f(t) e^{-i\lambda t}\,dt
$$
\end{proof}
\begin{remark}
        Как следствие, получаем более занятную формулу:
$$
        \mbox{Пусть } f \in \mathrm{C}^{k-1}(-\infty,\,+\infty),\quad \exists f^{(k)}:f^{(k)}\in\mathrm{L}_1(-\infty,\,+\infty), \mbox{ тогда} 
$$
$$
        F[f](\lambda) \leqslant \frac{C_m}{|\lambda|^m},\quad \mbox{где } C_m = \int\limits_{-\infty}^{+\infty}|f^{(k)}(t)|\,dt.
$$
\end{remark}
\begin{theorem}
        Пусть задана функция $f$ такая, что $\int_{-\infty}^t f(s)\,ds \in \mathrm{L}_1(-\infty,\,+\infty)$, тогда
$$
        F\left[
\int_{-\infty}^t f(s)\,ds
        \right] (\lambda)
=
        \frac1{i\lambda}F[f](\lambda).
$$
\end{theorem}
\begin{theorem}
        Пусть задана функция $f$ такая, что $t\cdot f(t) \in \mathrm{L}_1(-\infty,\,+\infty)$, тогда
$$
        F[f]'(\lambda) = F[-it\cdot f(t)](\lambda).
$$
\end{theorem}
\begin{proof}
$$
        \left(
\int\limits_{-\infty}^{+\infty}f(t)e^{-i\lambda t}\,dt
        \right)'_{\lambda}
=
        \int\limits_{-\infty}^{+\infty} (-it)f(t)e^{-i\lambda t}\,dt.
$$
\end{proof}
\begin{remark}
        Как следствие:
$$
        \mbox{Пусть } f : t^pf(t)\in \mathrm{L}_1(-\infty,\,+\infty), \; p = \overline{1, k}, \mbox{ тогда}
$$
$$
        F[f]^{(k)}(\lambda) = F[(-it)^k\cdot f(t)].
$$
\end{remark}
\begin{theorem}
        Пусть $t^p f(t) \in \mathrm{L}_1(-\infty,\,+\infty)\;\forall p$, тогда
$$
        F\left[
-\frac{1}{it}f(t)
        \right] (\lambda)
=
        \int\limits_{-\infty}^{\lambda} F[f](\xi)\,d\xi.
$$
\end{theorem}

Теперь поговорим о свойствах преобразования Фурье, связанных с операцие свертки. Напомним, как выглядит эта операция:
$$
        (f_1 * f_2)(t) = \int\limits_{-\infty}^{+\infty}f_1(s)f_2(t-s)ds.
$$
Эта операция является билинейной, коммутативной и ассоциативной.